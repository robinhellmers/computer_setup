\subsection{VirtualBox Extra Setup}
\subsubsection{Full-screen}
\begin{enumerate}
    \item Start virtual machine.
    \item Press \texttt{Devices} drop down list in the virtual box window. That is, not inside the virtual machine itself.
    \item Press \texttt{Insert Guest Additions CD image...}
    \item A popup in the virtual machine should show: \texttt{... contains software intended to be automatically started. Would you like to run it?} and choose \texttt{Run}.
    \item Resize the window a little and it will be full-screen.
\end{enumerate}

\subsubsection{Shared clipboard}
\begin{enumerate}
    \item In virtual box settings go to \texttt{General->Advanced} and select \texttt{Bidirectional} for \texttt{Shared Clipboard:} 
    \item Start virtual machine and see if it is working. If not, continue
    \item Press \texttt{Devices} drop down list in the virtual box window. That is, not inside the virtual machine itself. Press \texttt{Insert Guest Additions CD image...}
    \item If an error occurs do the following and then redo it
    \begin{itemize}
        \item Unmount VBoxGuestAdditons by \texttt{Devices->Optical Drives->Remove disk from virtual drive}.
    \end{itemize}
    \item Reboot the virtual machine.
    \item If it is not working, continue
    \item Download and install \textit{Extension pack} from virtual box.
    \item Reboot the virtual machine.
    \item If it is not working, try unmount and mount guest additions again.
\end{enumerate}

\subsubsection{Network setup for server and client IPv4 addresses}

When starting virtual machine: \texttt{Ctrl + Alt + T} for terminal.\\
Write: \texttt{ip addr show}, check wether ip-address is something like \texttt{192.11.1.24} and not \texttt{10.0.1.1}.\\
If something with \texttt{10.(...)}, then it is a local IPv4 address and not one from the DHCP of the router. 

Solution: Turn off virtual machine. Go to \(\rightarrow\) Settings \(\rightarrow\) Network and in \texttt{Attached to:} choose \texttt{Bridged Adapter} instead of probably NAT.

Start virtual machine and check if IPv4 address have changed to something like \texttt{192.(...)}.\\
If you open up a web-browser and don't get a connection, more settings have to be changed.\\
This probably depends on the virtual machine giving the router one MAC address and the host computer giving another.

Solution: Turn off virtual machine. Go to \(\rightarrow\) Settings \(\rightarrow\) Network and expand \texttt{Advanced}. Remove the MAC address. Then go to the host computer in Windows 10 and open \texttt{CMD}. Write: \texttt{ipconfig /all} and look for the MAC address of the host machine, probably named something like \\\code{Physical Address .............................................. 2C-F0-AF-73-2A-6C}\\
Input this instead of the old removed MAC address and save. This probably makes you unable to use internet on the host machine instead which one will have to sacrifice.


