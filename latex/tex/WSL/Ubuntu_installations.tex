\subsection{WSL Ubuntu installations} \label{sec:wsl_ubuntu_installations}

Start with:

\code{sudo apt update}

% --------------------------------------------------------------
% --------------------------------------------------------------
% --------------------------------------------------------------
\subsubsection{Terminal bookmark directories}
Install \texttt{Apparix} (Doc. \url{https://www.micans.org/apparix/}) with\\
\code{sudo apt-get install apparix}

Run \code{apparix} in order for it to set up its folders.

Then write \code{apparix --shell-examples} and copy everything except the aliases at the bottom. Paste this in \texttt{~/.bashrc}

If just copying from the terminal and pasting into \texttt{~/.bashrc} doesn't work, create a file and let the output be written into that instead in order to copy from it.

\code{touch text.txt}

\code{apparix --shell-examples > text.txt}

Then paste it into \texttt{~/.bashrc}.

Restart console.

Bookmark current directory with \code{bm bookmarkname} and go to the same location with \code{to bookmarkname}

% --------------------------------------------------------------
% --------------------------------------------------------------
% --------------------------------------------------------------
\subsubsection{Compiler \texttt{gcc} \& \texttt{g++}}
Compiler gcc and g++ installation:

\code{sudo apt install build-essential}

% --------------------------------------------------------------
% --------------------------------------------------------------
% --------------------------------------------------------------
\subsubsection{Debugger \texttt{gdb}}

\code{sudo apt install gdb}

% --------------------------------------------------------------
% --------------------------------------------------------------
% --------------------------------------------------------------
\subsubsection{Git}
Git installation:

\code{sudo apt install git}

Add credentials that will be asked for later on if not done:

\code{git config --global user.email "your@email.com"}

\code{git config --global user.name "your name"}

Initialize local repository:

\code{git init}

\code{git remote add origin <remote-address>}

Save credentials:

\code{git config credential.helper store}

Get master from remote origin:

\code{git pull origin master}

In order to not have to specify \code{<remote>} and \code{<branch>} in \code{git pull <remote> <branch>}, but still have to do previous pull first:

\code{git branch --set-upstream-to=origin/master master}

\code{git pull}

In order to pull another branch on the remote, firstly fetch the branches:

\code{git fetch}

Then see which branches that are fetched e.g.

\code{* [new branch]\hspace{1.5cm}testing\hspace{0.5cm}-> origin/testing}

Checkout that branch. OBS that you don't have a branch created locally, you will only be checking out the remote branch which may change as soon as you \code{git fetch} again. This is only in order to copy the content of the remote branch into the local one.

\code{git checkout origin/testing}

Create the local copy of the remote branch and go to it:

\code{git checkout -b testing}

Make sure that the local branch have its upstream to the remote one:

\code{git branch --set-upstream-to=origin/testing}

\subsubsubsection{Git log - .gitconfig}

In order to have a good git history tree visualization in the terminal, use the \texttt{.gitignore} from \href{https://github.com/robinhellmers/computer_setup/blob/master/git_setup/gitconfig}{Github}. This gives three different commands:\\
\code{git lg}\\
\code{git lg2}\\
\code{git lg3}

\subsubsection{LaTeX}

Install the complete version TeX Live:

\code{sudo apt install texlive-full}

\subsubsubsection{Minted - Code}
In order to use the \texttt{minted} package for displaying code one must install \texttt{Pygments} for \texttt{Python}.

If using \texttt{Ubuntu 18.04}:

\code{sudo apt install python-pygments}

If using \texttt{Ubuntu 20.04}:

\code{sudo apt install python3-pygments}

Also, one must use the \code{-shell-escape} flag with the \code{latexmk} (standard compilation) command in order to compile with \texttt{minted}.

In Visual Studio Code, press \texttt{Ctrl + Shift + P} in order to search for commands. Search and execute \code{Preferences: Open settings (JSON)} in order to open up the \texttt{settings.json} file with the compilation recipe. Add \code{"-shell-escape"} to the arguments.

If the file is empty, use the \texttt{settings.json} from \href{https://github.com/robinhellmers/computer_setup/blob/master/latex-workshop/settings.json}{Github}.

\begin{minted}[tabsize=3,obeytabs,linenos,bgcolor=codegray]{json}
{
    "name": "latexmk",
    "command": "latexmk",
    "args": [
        "-shell-escape",
        "-synctex=1",
        "-interaction=nonstopmode",
        "-file-line-error",
        "-pdf",
        "-outdir=%OUTDIR%",
        "%DOC%"
    ],
    "env": {}
}
\end{minted}

If there there is an error message similar to \code{Undefined control sequence. \textbackslash PYG \#1\#2->\textbackslash FV\(@\)PYG},\\
add the argument \code{cache=false} when loading the \code{minted} package. 

\code{\textbackslash usepackage[cache=false]\{minted\}}

\subsubsubsection{Visually continue row on new line}

Text visually continue on new line if row is too long, edit the shortcut key for\\\code{View: Toggle Word Wrap} in \texttt{File->Preferences->Keyboard Shortcuts}

\subsubsubsection{Glossaries}

In order for glossaries to work, one must call \code{makeglossaries}.

E.g. \code{pdflatex <file>.tex}  \(\rightarrow\) \code{makeglossaries <file>} \(\rightarrow\) \code{pdflatex <file>.text}

But with \texttt{latex-workshop} in \texttt{VSCode}, the standard compilation tool is \code{latexmk} which does a series of calls with e.g. \code{pdflatex}, \code{bibtex}, etc.

In order to make \code{latexmk} include a call of \code{makeglossaries}, create the file \code{~/.latexmkrc} from \href{https://github.com/robinhellmers/computer_setup}{Github}.

\begin{minted}[tabsize=3,obeytabs,bgcolor=codegray]{shell}
add_cus_dep('glo', 'gls', 0, 'makeglo2gls');
add_cus_dep('acn', 'acr', 0, 'makeglo2gls');
sub makeglo2gls {
        system("makeglossaries $_[0]");
}
\end{minted}